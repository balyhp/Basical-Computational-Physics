\documentclass[12pt,a4paper]{article}
%\usepackage{ctex}
\usepackage{amsmath}
\usepackage{amssymb}
\usepackage{graphicx}
\usepackage{geometry}
\usepackage{listings}
\usepackage{xcolor}
\usepackage{float}

\geometry{left=2.5cm,right=2.5cm,top=2.5cm,bottom=2.5cm}

\lstset{
    language=Python,
    basicstyle=\ttfamily\small,
    keywordstyle=\color{blue},
    commentstyle=\color{gray},
    stringstyle=\color{red},
    numbers=left,
    numberstyle=\tiny,
    stepnumber=1,
    numbersep=5pt,
    frame=single,
    breaklines=true,
    backgroundcolor=\color{white},
    tabsize=4,
    captionpos=b
}

\title{Homework 1}
\author{zric \qquad 123456789}

\date{\today}

\begin{document}

\maketitle
\begin{center}
Note: This template is for course \texttt{ basic computational physics} homework assignments. Please fill in your name and student ID above. 
\end{center}

\section{Problem1:\qquad}
\subsection{problem description}
state the problem here

\subsection{algorithm description}
talk about the algorithm here, say monte carlo, finite difference, etc.

\subsection{pseudocode} 
\begin{lstlisting}[language={}, frame=lines]
1. initialize variables
2. input
3. main loop
4. output
\end{lstlisting}


\section{Problem2:\qquad}
\subsection{problem description}

\subsection{ algorithm description}


\subsection{pseudocode}
 
\begin{lstlisting}[language={}, frame=lines]
1. initialize variables
2. input
3. main loop
4. output
\end{lstlisting}


\section{output examples}
\subsection{problem1}
show screenshots or output examples here
\subsection{problem2}

\end{document}