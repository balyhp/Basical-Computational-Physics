\documentclass[12pt,a4paper]{article}
%\usepackage{ctex}
\usepackage{amsmath}
\usepackage{amssymb}
\usepackage{graphicx}
\usepackage{geometry}
\usepackage{listings}
\usepackage{xcolor}
\usepackage{float}
\usepackage[english]{babel}

\geometry{left=2.5cm,right=2.5cm,top=2.5cm,bottom=2.5cm}

\lstset{
    language=Python,
    basicstyle=\ttfamily\small,
    keywordstyle=\color{blue},
    commentstyle=\color{gray},
    stringstyle=\color{red},
    numbers=left,
    numberstyle=\tiny,
    stepnumber=1,
    numbersep=5pt,
    frame=single,
    breaklines=true,
    backgroundcolor=\color{white},
    tabsize=4,
    captionpos=b
}

\title{Homework4}
\author{Zric \qquad 23307110123}
\date{\today}

\begin{document}

\maketitle
\section{Problem1:\quad Interpolation}
\subsection{problem description}
Newton interpolation of 

(i) 10 equal spacing points of $\cos (x)$ within [0,$\pi$],

(ii) 10 equal spacing points $\displaystyle \frac{1}{1+25x^2}$ within [-1,1]. 

Compare the results with the cubic spline interpolation. 
\subsection{algorithm description}
\noindent(1)Newton interpolation:

Given data points $(x_0, y_0), (x_1, y_1), \ldots, (x_n, y_n)$, the Newton interpolation polynomial is constructed using divided differences. The polynomial $P(x)$ is given by:
\begin{equation}
P(x) = f[x_0] + f[x_0, x_1](x - x_0) + \ldots  + f[x_0, x_1, \ldots, x_n](x - x_0)(x - x_1) \ldots (x - x_{n-1}) 
\end{equation}
where $f[x_i, x_{i+1}, \ldots, x_j]$ are the divided differences.
can be computed recursively as:
\begin{equation}
f[x_0] = y_0, \quad f[x_0, x_{1}, \ldots, x_i] = \frac{f[x_{1}, \ldots, x_i] - f[x_0, \ldots, x_{i-1}]}{x_i - x_0},(i=1,2,...,n)
\end{equation}

\noindent(2)Cubic Spline Interpolation:

Given data points $(x_0, y_0), (x_1, y_1), \ldots, (x_n, y_n)$, cubic spline interpolation compute $n$ cubic polynomials $f_i(x)=a_ix^3+b_ix^2+c_ix+d_i$ for each interval $[x_i, x_{i+1}], i=0,1,2...,n-1$.

The function $f_i(x)$ must satisfy the following conditions:
\begin{enumerate}
    \item $f_i(x_{i})=y_i; f_{i}(x_{i+1})=y_{i+1}$ , for $i = 0, 1, \ldots, n-1$ (giving 2n equations).
    \item $f'_i(x_{i+1}) = f'_{i+1}(x_{i+1})$ for $i = 0, 1, \ldots, n-2$. (giving n-1 equations).
    \item $f''_i(x_{i+1}) = f''_{i+1}(x_{i+1})$ for $i = 0, 1, \ldots, n-2$. (giving n-1 equations).
    \item $f''_{0}(x_0)=f''_{n-1}(x_n)=0$ (giving 2 equations).so there are 4n equations in total.
\end{enumerate}

Let $h_i = x_{i+1} - x_i$. The cubic polynomial for the interval $[x_i, x_{i+1}]$ can be written as:
\begin{equation}
f_i(x) = \frac{f''(x_i)}{6h_i}(x_{i+1}-x)^3 + \frac{f''(x_{i+1})}{6h_i}(x-x_i)^3 + \left(\frac{y_{i+1}}{h_i} - \frac{f''(x_{i+1})h_i}{6}\right)(x-x_i) + \left(\frac{y_i}{h_i} - \frac{f''(x_i)h_i}{6}\right)(x_{i+1}-x)
\end{equation}

The continuity of the first derivative $f'_{i-1}(x_i) = f'_i(x_i)$ leads to a system of linear equations for the unknown second derivatives $f''(x_i)$:
\begin{equation}
\frac{h_{i-1}}{6}f''(x_{i-1}) + \frac{h_{i-1}+h_i}{3}f''(x_i) + \frac{h_i}{6}f''(x_{i+1}) = \frac{y_{i+1}-y_i}{h_i} - \frac{y_i-y_{i-1}}{h_{i-1}}
\end{equation}
for $i = 1, 2, \ldots, n-1$.This gives $n-1$ equations for $n+1$ unknown second derivatives $f''(x_i)$. Plus 2 natural conditions $ f''_{0}(x_0)=f''_{n-1}(x_n)=0$, and it's able to be solved in a tridiagonal matrix with Thomas algorithm, which I employed in my code, this algorithm is simlilar to gaussian elimination, forward elimination to get upper triangluar matrix plus backward solving, but only acting on 3 diagonal elements every step, with a time complexity of O(n). 

\subsection{output}
run \texttt{problem1.py}
\begin{figure}[H]
    \centering
    \includegraphics[width=0.6\linewidth]{pic/case1_cos.png}
    \caption{$\cos(x) $ interpolation}
\end{figure}
The Newton interpolation method agree well with Cubic spline method here, both in good agreement with the true value of $\cos(x)$, this is because $\cos(x)$ can be expanded in power series $\sum_{k=0}^{\infty}\frac{(-)^k}{(2k)!}x^{2k}$ , and using the first low order terms can describe this function precisely(because high order terms can be small enough to be ignored at 0 or $\pi$ , $\lim_{k \rightarrow \infty} \frac{\pi^k}{(k)!}=0$).
So the Newton polynomial didn't appear Runge error at boundary.
\begin{figure}[H]
    \centering
    \includegraphics[width=0.6\linewidth]{pic/case2_runge.png}
    \caption{$\frac{1}{1+25x^2}$ interpolation}
\end{figure}
In this case the cubic spline method achieves a better interpolation than Newton method, because cubic spline guarantee the smooth in each interval while Newton method appears Runge error at two ends.This is obvious when consider $\frac{1}{1+25x^2}=\sum_{k=0}^{\infty} (-)^k(25x^2)^k$, the high order terms cannot be ignored when x approaches $\pm 1$.

\section{Problem2:\quad Least-square fit}
\subsection{problem description}
The table below gives the temperature $T$ along a metal rod whose ends are kept at fixed constant temperatures. The temperature $T$ is a function of the distance $x$ along the rod.
\begin{table}[H]
    \centering
    \begin{tabular}{|c|c|c|c|c|c|c|c|c|c|}
        \hline
        $x$ (cm) & 1.0 & 2.0 & 3.0 & 4.0 & 5.0 & 6.0 & 7.0 &8.0& 9.0  \\
        \hline
        $T$ (°C) & 14.6 & 18.5 & 36.6 & 30.8 & 59.2 & 60.1& 62.2 &79.4 & 99.9 \\
        \hline
    \end{tabular}
\end{table}

(i) Compute a least-squares, straight-line fit to these data using $T(x)=ax+b$

(ii) Compute a least-squares, parabolic-line fit to these data using$T(x)=ax^2+bx+c$

\subsection{algorithm description}
For straight-line fit , target function is 
\begin{equation}
    T(a,b)=\sum_i [y_i-(ax_i+b)]^2
\end{equation}
with $\frac{\partial T}{\partial a}=0,\quad \frac{\partial T}{\partial b}=0$, we have:
\begin{equation}
    \begin{cases}
        &a\sum_i x_i^2+b\sum_i x_i =\sum_i x_i y_i\\
        &a\sum_i x_i + b\sum_i 1 =\sum_i y_i
    \end{cases}
\end{equation} 
consider it as a $2\times 2$ matrix $Ax=b$, I use gaussian elimination method to get $x=[a,b]^T$.

For parabolic-line fit, target function is:
\begin{equation}
    T(a,b,c)=\sum_i [y_i-(ax_i^2+bx_i+c)]^2
\end{equation}
with $\frac{\partial T}{\partial a}=0,\quad \frac{\partial T}{\partial b}=0,\quad \frac{\partial T}{\partial c}=0$, we have:
\begin{equation}
    \begin{cases}
        &a\sum_i x_i^4 + b\sum_i x_i^3 + c\sum_i x_i^2 = \sum_i x_i^2 y_i\\
        &a\sum_i x_i^3 + b\sum_i x_i^2 + c\sum_i x_i = \sum_i x_i y_i\\
        &a\sum_i x_i^2 + b\sum_i x_i + c\sum_i 1 = \sum_i y_i
    \end{cases}
\end{equation}
Simlilarly, consider it as a $3\times 3$ matrix $Ax=b$, and use gaussian elimination method to get $x=[a,b,c]^T$.
\subsection{output}
run \texttt{problem2.py}
\begin{figure}[H]
    \centering
    \includegraphics[width=0.6\linewidth]{pic/problem2_fit.png}
    \caption{Least-square fit results}
\end{figure}
the fitting results show that the parabolic fit is better than linear fit, which can be seen from the Figure 4 below, linear fit's RMSE=6.5063, R$^2$=0.9411; parabolic fit's RMSE=6.066, R$^2$=0.9488.
\begin{figure}[H]
    \centering
    \includegraphics[width=0.6\linewidth]{pic/problem2.png}
    \caption{Residuals of least-square fit}
\end{figure}

\end{document}