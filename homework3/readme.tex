\documentclass[12pt,a4paper]{article}
%\usepackage{ctex}
\usepackage{amsmath}
\usepackage{amssymb}
\usepackage{graphicx}
\usepackage{geometry}
\usepackage{listings}
\usepackage{xcolor}
\usepackage{float}
\usepackage[english]{babel}

\geometry{left=2.5cm,right=2.5cm,top=2.5cm,bottom=2.5cm}

\lstset{
    language=Python,
    basicstyle=\ttfamily\small,
    keywordstyle=\color{blue},
    commentstyle=\color{gray},
    stringstyle=\color{red},
    numbers=left,
    numberstyle=\tiny,
    stepnumber=1,
    numbersep=5pt,
    frame=single,
    breaklines=true,
    backgroundcolor=\color{white},
    tabsize=4,
    captionpos=b
}

\title{Homework3}
\author{Zric}
\date{\today}

\begin{document}

\maketitle
\section{Problem1:\quad LU algorithm}
\subsection{problem description}
Prove that the time complexity of the LU decomposition algorithm algorithm is $O(n^3)$,here L means lower triangular matrix and U upper triangular matrix.

\subsection{solution}
Proof:

For a nonsingular matrix $A\in\mathbb{R}^{n\times n}$ (with partial pivoting if needed), there exist a permutation matrix $P$, a unit lower triangular matrix $L$, and an upper triangular matrix $U$ such that
\begin{equation}
    PA = LU
\end{equation}
The upper and lower triangular matrix satisfies:
\begin{equation}
    a_{ij}=\sum_{s=1}^{n} l_{is}u_{sj},\qquad l_{ii}=1,\; l_{ij}=0\ (j>i),\; u_{ij}=0\ (i>j)
\end{equation}
(1)Time complexity of LU decomposition:
the decomposition can be devided into 3 steps:

step1: get the first column of $L$ and the first row of $U$
\begin{equation}
    \begin{cases}
        &l_{i1}=a_{i1} \qquad i=1,2..n\\
        &u_{1j}=\frac{a_{1j}}{l_{11}} \qquad j=1,2..n\\
    \end{cases}
\end{equation}

step2: get the second and following elements of $L$
\begin{equation}
       l_{ij}=a_{ij}-\sum_{k=1}^{j-1}l_{ik}u_{kj} \qquad j=2,3,...n \quad i=j,j+1,...n
\end{equation}

step3: use results from step2 to get elements of $U$ 
\begin{equation}
    u_{ji} = \frac{1}{l_{jj}}(a_{ji}-\sum_{k=1}^{j-1} l_{jk}u_{ki}) \qquad j=2,3,...n-1 \quad i=j+1,j+2,...n
\end{equation}
In the whole loop ,step2 and step3 is coupled with each other, each will use the output elements of the other one.The first step has complexity of $O(n)$. And in step2, the total number of loop is
 \begin{equation}
  \sum_{j=2}^{n} (n-j+1)(j-1) = \frac{1}{2}n^2(n-1)-\frac{1}{6}n(n-1)(2n-1)=\frac{1}{6}n^3-\frac{1}{6}n
\end{equation}
The same applies to step3, total number of loop in step3:
\begin{equation}
    \sum_{j=2}^{n-1} (n-j)(j-1)=\frac{1}{6}n^3-\frac{1}{2}n^2+\frac{1}{3}n
\end{equation}
(2)Time complexity of forward and backward substitution:

In solving $LUx=b$, forward substitution means solving $Ly=b$, backward substitution means solving $Ux=y$, the two process has the same time complexity $O(n^2)$, for a lower or upper triangular matrix, the total number of loop is:
\begin{equation}
    \sum_{j=1}^{n}(j-1)=\frac{n(n-1)}{2}
\end{equation} 

Sum over equation(6)(7)(8), the leading term being $\frac{1}{3}n^3$,  hence the LU decomposition algorithm has time complexity $O(n^3)$.

\section{Problem2:\quad }
\subsection{problem description}
Solve systems of equations using the Gaussian elimination algorithm and partial-pivoting scheme (Note: Write a general program applicable to solving the following equations; select the pivot element with the largest coefficient among all columns for each elimination step; Optional: Compute the inverse matrix of the coefficient matrix).

\begin{equation}
    \begin{cases}
2x_1 + 3x_2 + 5x_3 &= 5,\\
 3x_1 + 4x_2 + 8x_3 &= 6,\\
 x_1 + 3x_2 + 3x_3 &= 5.
    \end{cases}
\end{equation}

\subsection{algorithm description}
We solve $A x = B$ by Gaussian elimination with partial pivoting on the augmented matrix $[A\;|\;B]$, here $B$ can be generalized as $B\in\mathbb{R}^{n\times m}$. 

(1) Forward elimination (with partial pivoting):

First for $k=1,\dots,n$, Select the pivot row $p=\max_{i\ge k}|a_{ik}|$ in column $k$ and swap rows $k\leftrightarrow p$ in the whole augmented matrix.

Second for each $i=k+1,\dots,n$, form the multiplier $l_{ik}=a_{ik}/a_{kk}$, then apply the row operation to each row below k
\begin{equation}
    \text{row}_i \leftarrow \text{row}_i - l_{ik}\,\text{row}_k,
\end{equation}
    
updating both the left block (turning $A$ into an upper triangular $U$) and the right block $B$ simultaneously.
After this phase, the left block is $U$, and the right block is $B' = L^{-1}B$ (forward substitution has been implicitly applied by the same row operations).

(2) Back substitution (simultaneous for all right-hand sides):
Solve $U x = B'$ from bottom to top. For $i=n,\dots,1$ and for each RHS column $r$,
\begin{equation}
    x_{ir}=\frac{1}{u_{ii}}\Big(b'_{ir}-\sum_{j=i+1}^{n}u_{ij}x_{jr}\Big).
\end{equation}

(3) Matrix inverse (optional):

To get inversion matrix of $A$ ,just set $B=I_n$ as identity matrix, and run the same procedure. The result $x$ equals $A^{-1}$.

(4)Time complexity
Let $A\in\mathbb{R}^{n\times n}$ and $B\in\mathbb{R}^{n\times m}$.

Pivot search: $\sum_{k=1}^{n}(n-k+1)=O(n^2)$ comparisons.

Updating the left block (turning $A$ into $U$): 
$2\sum_{k=1}^{n-1}(n-k)^2=\frac{2}{3}n^3+O(n^2)$ flops.

Updating the right block during elimination (equivalent to forward substitution for $n\times m$ RHS):  about $m n^2+O(mn)$ flops.

Back substitution for $n\times m$ RHS: about $m n^2+O(mn)$ flops.

Total: $\text{flops}=\frac{2}{3}n^3 + 2 m n^2 + O(n^2+mn) = O(n^3).$
\subsection{output}
\begin{figure}[H]
    \centering
    \includegraphics[width = 0.60\linewidth]{pic/q2.png}
    \caption{output of problem2.py}
\end{figure}

\section{Problem3:\quad  }
\subsection{problem description}
Solve the 1D Schrodinger equation with the potential (i) $V(x)=x^2$
; (ii) $V(x)=x^4-x^2$ with the variational approach using a Gaussian basis (either fixed widths or fixed centers). Consider the three lowest energy eigenstates.

The Gaussian basis functions are defined as: $\displaystyle \phi_i(x)=(\frac{v_i}{\pi})^{1/2}\exp(-v_i(x-s_i)^2)$
. This function has two variational parameters: 
$v_i$ the width of the Gaussian, and $s_i$ the center of the Gaussian. For simplicity, we only vary one of these parameters at a time and do calculations with either fixed widths or fixed centers. 
\subsection{algorithm description}
To solve the one-dimensional Schrödinger equation in Hatree units
\begin{equation}
    \hat{H}\psi(x) = E\psi(x), \qquad 
\hat{H} = -\tfrac{1}{2}\frac{d^2}{dx^2} + V(x),
\end{equation}
with the potentials $V(x)=x^2$ and $V(x)=x^4 - x^2$ 

The trial wave function is expanded as
\begin{equation}
    \psi(x) = \sum_i c_i\,\phi_i(x), \qquad
\phi_i(x) = \sqrt{\tfrac{v_i}{\pi}}\,e^{-v_i(x-s_i)^2},
\end{equation}
where \(v_i\) is the Gaussian width and \(s_i\) its center.

For each pair of basis functions, analytical integrals are used to build:
\begin{equation}
    S_{ij} = \langle\phi_i|\phi_j\rangle, \qquad
T_{ij} = \tfrac{1}{2}\!\int \phi_i'(x)\phi_j'(x)\,dx, \qquad
V_{ij} = \langle\phi_i|V(x)|\phi_j\rangle.
\end{equation}

The Hamiltonian matrix is \(H = T + V\).
Because the basis is non-orthogonal (\(S\neq I\)),the eigenvalue equation can be write in matrix form.$c= [c_1,c_2,...,c_n]^T$
\begin{equation}
    Hc = \varepsilon S c
\end{equation}

S can be decomposed using Cholesky method as \(S=R^{T}R\), $R$ being the upper triangular matrix.Multiply $R^{-T}$ on the left side:
\begin{equation}
   R^{-T}HR^{-1}Rc = \varepsilon R^{-T}R^{T}Rc \rightarrow   A y = \varepsilon y, \quad A = R^{-T} H R^{-1}, \quad y = Rc
\end{equation}

So we just need to find eigenvalues of $A$ . In the code I use fixed-center basis, I take 9 Gaussian functions and set their centers in [-3,3] with the same interval, and they share the same width $v$ for simplicity. 

Then a grid of width \(v\in[0.2,3.0]\) is scanned.
The value minimizing the ground-state energy \(E_0(v)\)
is chosen as the optimal width \(v_{\text{best}}\) to get $E_0$. Then the second and third excitation energies are computed in the same way. 

\subsection{output}
\begin{figure}[H]
    \centering
    \includegraphics[width=0.8\linewidth]{pic/q3.png}
    \caption{output of problem3.py}
\end{figure}
the output includes the best width for each energy.
\end{document}