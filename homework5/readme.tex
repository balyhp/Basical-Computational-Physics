\documentclass[12pt,a4paper]{article}
%\usepackage{ctex}
\usepackage{amsmath}
\usepackage{amssymb}
\usepackage{graphicx}
\usepackage{geometry}
\usepackage{listings}
\usepackage{xcolor}
\usepackage{float}
\usepackage[english]{babel}

\geometry{left=2.5cm,right=2.5cm,top=2.5cm,bottom=2.5cm}

\lstset{
    language=Python,
    basicstyle=\ttfamily\small,
    keywordstyle=\color{blue},
    commentstyle=\color{gray},
    stringstyle=\color{red},
    numbers=left,
    numberstyle=\tiny,
    stepnumber=1,
    numbersep=5pt,
    frame=single,
    breaklines=true,
    backgroundcolor=\color{white},
    tabsize=4,
    captionpos=b
}

\title{Homework5}
\author{Zric}
\date{\today}

\begin{document}

\maketitle
\section{Problem1:\quad five-point formula for $f''(x)$}
\subsection{problem description}
Derive the five-point formula for the second-order derivative $f''(x)$.

\subsection{Proof}
with centered difference method, we have:
\begin{equation}
    \begin{cases}
        f(x+2h)+f(x-2h)-2f(x)&=4h^2f''(x)+\frac{16h^4}{24}f^{(4)}(x)+O(h^6)\\
        f(x+h)+f(x-h)-2f(x)&=h^2f''(x)+\frac{h^4}{24}f^{(4)}(x)+O(h^6)\\
    \end{cases}
\end{equation}
In order to eliminate $f^{(4)}(x)$, we multiply the second equation by -16 and add it to the first equation,which gives:
\begin{equation}
    f(x+2h)-16f(x+h)+30f(x)-16f(x-h)+f(x-2h)=-12h^2f''(x)+O(h^6)
\end{equation}
so we have the five-point formula for the second-order derivative:
\begin{equation}
    f''(x)=\frac{-f(x+2h)+16f(x+h)-30f(x)+16f(x-h)-f(x-2h)}{12h^2}+O(h^4)
\end{equation}

\section{Problem2:\quad Romberg Integration}
\subsection{problem description}
Compute integration of $f(x)=\exp(-x^2)$ from 0 to 1 using Romberg Integration, using at least 4 layers of extrapolation to compute and analyze the error.
\subsection{algorithm description}
Romberg integration  computes a sequence of trapezoidal approximations with equal $2^k$ subdivisions of the integration interval, and then use Richardson extrapolation to eliminate leading-order error terms.
\begin{equation}
    \int_{a}^{b}dx f(x) \approx R(i,0)=\frac{b-a}{2^{i+1}}\left[f(a)+f(b)+2\sum_{n=1}^{2^i-1}f\left(a+n\frac{b-a}{2^i}\right)\right]
\end{equation}
and we can compute higher-order estimates recursively :
\begin{equation}
    R(i,k)=\frac{4^k R(i,k-1)-R(i-1,k-1)}{4^k-1}
\end{equation}
where $R(i,k)$ is the k-th extrapolated value at level i.
\subsection{output}
run \texttt{problem2.py}
\begin{figure}[H]
    \centering
    \includegraphics[width=0.9\textwidth]{pic/p1.png}
    \caption{output of Romberg Integration}
\end{figure}

\section{Problem3:\quad }
\subsection{problem description}
Radial wave function of the 3s orbital is: 
\begin{equation}
    R_{3s}(r) = \frac{1}{9\sqrt{3}}\,(6 - 6\rho + \rho^{2})\,Z^{3/2}\,e^{-\rho/2}
\end{equation}
\begin{enumerate}
    \item r = radius expressed in atomic units (1 Bohr radius = 52.9 pm)
    \item e = 2.71828 approximately.
    \item Z = effective nuclear charge for that orbital in that atom.
    \item $\rho$= $2\frac{Zr}{n}$, where n is the principal quantum number (3 for the 3s orbital)
\end{enumerate}

Compute $\displaystyle \int_{0}^{40}|R_{3s}(r)|^2r^2dr$  for Si atom (Z=14) with Simpson's rule using two different radial grids:

(1)Equal spacing grids: $r[i]=(i-1)h$; i = 1,...,N (try different N)

(2)A nonuniform integration grid, more finely spaced at small r than at large r: $r[i] = r_0 (e^{t[i]}-1)$; $t[i]=(i-1)h$; i = 1, ..., N (One typically choose $r_0$ = 0.0005 a.u., try different N).

(3)Find out which one is more efficient, and discuss the reason.

\subsection{algorithm description}
(1) For uniform grid $r[i]=(i-1)h$; i = 1,...,N , we use Simpson's rule, so N must be odd:
\begin{equation}
    \int_{a}^{b}f(x)dx \approx \frac{h}{3}\sum_{i=1}^{N-2}\left[f(x_i)+4f(x_{i+1})+f(x_{i+2})\right],\quad h=\frac{b-a}{N-1}
\end{equation}
we can thus compute the integral directly.

(2) For non-uniform grid $r[i] = r_0 (e^{t[i]}-1)$; $t[i]=(i-1)h$; i = 1, ..., N , notice that $t[i]=(i-1)h$ is uniform, we can use the change of variable to convert the integral:
\begin{equation}
    \int_{0}^{40}|R_{3s}(r)|^2r^2dr = \int_{t(1)}^{t(N)}|R_{3s}(r(t))|^2r(t)^2\frac{dr}{dt}dt
\end{equation}
where $\displaystyle \frac{dr}{dt} = r_0 e^{t}$. we can then use Simpson's rule (equation (7)) to compute the integral over t, since t is a uniform grid.
\begin{figure}[H]
    \centering
    \includegraphics[width=0.6\textwidth]{pic/p3.png}
    \caption{Integral function $|R_{3s}(r)|^2r^2$}
\end{figure}

\subsection{output}
run \texttt{problem3.py}
\begin{figure}[H]
    \centering
    \includegraphics[width=0.8\textwidth]{pic/p3output.png}
    \caption{output of uniform and non-uniform grid integration}
\end{figure}

Analysis: By comparing the results of the two methods with different N, we can find out nonuniform method is more efficient.

This is because the radial wave function changes rapidly at small r in [0,2] (shown in Figure 2) and slowly at large r, so using a nonuniform grid that is more finely spaced at small r can capture the behavior of the function more accurately, leading to better integration results than uniform grid with the same N points.
\end{document}